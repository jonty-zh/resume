% !TEX TS-program = xelatex
% !TEX encoding = UTF-8 Unicode
% !Mode:: "TeX:UTF-8"

\documentclass{resume}
\usepackage{zh_CN-Adobefonts_external} % Simplified Chinese Support using external fonts (./fonts/zh_CN-Adobe/)
% \usepackage{NotoSansSC_external}
% \usepackage{NotoSerifCJKsc_external}
% \usepackage{zh_CN-Adobefonts_internal} % Simplified Chinese Support using system fonts
\usepackage{linespacing_fix} % disable extra space before next section
% \usepackage{cite}
% \usepackage[colorlinks,linkcolor=blue]{hyperref}

\begin{document}
\pagenumbering{gobble} % suppress displaying page number

\name{张云涛}

\basicInfo{
  \email{yuntaochn@163.com} \textperiodcentered\ 
  \phone{(+86) 183-5343-1432} \textperiodcentered\ 
  % \linkedin[billryan8]{https://www.linkedin.com/in/billryan8}
  }

\section{\faGraduationCap\  教育背景}
\datedsubsection{\textbf{哈尔滨工程大学}}{2022 -- 至今}
\textit{在读硕士研究生}\quad 土木水利
\datedsubsection{\textbf{南京大学}}{2014 -- 2018}
\textit{学士}\quad 自动化

\section{\faUsers\ 工作/项目经历}
\datedsubsection{\textbf{埃夫特机器人公司}\quad 上海}{2018 -- 2022}
\role{研发工程师}{}
工业机器人相关软件研发
% xxx后端开发
% \begin{itemize}
%   \item 实现了 xxx 特性
%   \item 后台资源占用率减少8\%
%   \item xxx
% \end{itemize}

\datedsubsection{\textbf{机器人喷涂}}{2021 -- 2022}
\role{}{前后端开发,驱动开发,扫描软件}
\begin{onehalfspacing}
开发机器人自动喷涂产线软件,相机扫描工件,根据工件深度图生成喷涂工序,工业机械臂进行自动化喷涂。
\begin{itemize}
  \item 驱动开发,制定产线内多种硬件通信协议,继而通过PLC统一控制。
  % \item 任务管理,线程池管理生产任务
  \item 前后端开发,通过ORM建立数据库与内存映射,Qt开发前端界面,并通过其MVC架构联系前后端。
  \item 扫描软件开发,开发相机驱动,以及融合多相机数据,实现扫描工件,得到深度图像。
\end{itemize}
\end{onehalfspacing}

\datedsubsection{\textbf{ROS接口开发}}{2020 -- 2020}
\role{}{协作机器人与多种型号工业机器人ROS包开发}
\begin{onehalfspacing}
协作机器人可省去控制器,直连上位计算机,其关节控制基于CAN通信协议,基于CAN接口开发ROS包,以便于二次开发。

工业机器人基于socket或其私有通信协议开发ROS功能包。
\begin{itemize}
  \item CAN驱动开发
  \item ROS功能包,包括仿真模型,moveit,以及rviz包等。
  \item Qt开发上位机,实现上位控制机器人上下伺服、获取机器人状态信息等功能。
\end{itemize}
\end{onehalfspacing}

\datedsubsection{\textbf{机器人分拣}}{2018 -- 2019}
\role{}{驱动开发, 仿真}
\begin{onehalfspacing}
机器人自适应分拣项目,基于强化学习DQN算法进行训练,利用深度相机与机械臂,实现自动分拣,
以期应用于物流分拣等场景。
  \begin{itemize}
  \item 硬件控制,驱动开发,下位控制器程序开发,夹具驱动开发。
  \item 仿真场景搭建,VREP搭建仿真场景,通过redis以及ZeroMQ等传输数据。
\end{itemize}
\end{onehalfspacing}


% Reference Test
% \datedsubsection{\textbf{Paper Title\cite{zaharia2012resilient}}}{May. 2015}
% An xxx optimized for xxx\cite{verma2015large}
% \begin{itemize}
%  \item main contribution
% \end{itemize}

\section{\faHeartO\ 获奖情况}
\datedline{\textit{人民奖学金}\quad 三等奖}{2016}
\datedline{\textit{中国教育机器人大赛}\quad 特等奖}{2015}
\datedline{\textit{江苏省机器人大赛}\quad 二等奖}{2016}
\datedline{\textit{Robomaster}\quad 东部赛区三等奖}{2017}

\section{\faPencil\ 发表}
[1]翟昱,张云涛,马英,易廷昊,邹鹏. 基于点云几何特征的角焊缝定位方法[P]. 安徽省:CN113177983A,2021-07-27.  

[2]姜宇帆,易廷昊,翟昱,马英,张云涛. 基于竞争性的分布式调度系统及其调度方法[P]. 安徽省:CN112529260A,2021-03-19. 

[3]翟昱,马英,代夷帆,姜宇帆,易廷昊,张云涛. 一种全自动机器人手眼标定方法[P]. 安徽省:CN111152223A,2020-05-15. 

[4]易廷昊,翟昱,代夷帆,姜宇帆,张云涛,马英. 一种面向多目标种类的机械臂自适应抓取方法[P]. 安徽省:CN109986560A,2019-07-09. 


\section{\faCogs\ 相关技能}
% increase linespacing [parsep=0.5ex]
\begin{itemize}[parsep=0.5ex]
  \item 资格证书: CET-4 589、江苏省计算机三级偏硬
  \item 编程语言: C++、Python、Qt 
  \item 平台: Linux、Docker
  \item 机器人相关: ROS、仿真软件
  \item 工具: Latex、Git、CAD等  
\end{itemize}

% \section{\faInfo\ 其他}
% % increase linespacing [parsep=0.5ex]
% \begin{itemize}[parsep=0.5ex]
%   \item 资格证书
%   \item CET-4 589
%   \item 江苏省计算机三级偏硬
% \end{itemize}

% % Reference
% \newpage
% \bibliographystyle{IEEETran}
% \bibliography{mycite}
\end{document}
